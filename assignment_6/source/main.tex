\documentclass{article}
\usepackage[utf8]{inputenc}
\usepackage{enumitem}
\usepackage{karnaugh-map}
\usepackage[american]{circuitikz}

\title{DLD Assignment 6}
\author{Komireddy Lahari}

\begin{document}

\maketitle

\section{Question}
A function F(A,B,C) defined by three Boolean variables A, B and C when expressed as sum of products is given by
\begin{equation}
    F=\overline{A}.\overline{B}.\overline{C}+\overline{A}.B.\overline{C}+A.\overline{B}.\overline{C}
\end{equation}

where, $\overline{A},\overline{B}$, and $\overline{C}$ are the components of the respective variables. The product of sums (POS) form of the function F is

\begin{enumerate}[label=(\Alph*)]
    \item F=(A+B+C).(A+$\overline{B}$+C).($\overline{A}$+B+C)
    \item F=($\overline{A}$+$\overline{B}$+$\overline{C}$).($\overline{A}$+B+$\overline{C}$).(A+$\overline{B}$+$\overline{C}$)
    \item F=(A+B+$\overline{C}$).(A+$\overline{B}$+$\overline{C}$).($\overline{A}$+B+$\overline{C}$).($\overline{A}$+$\overline{B}$+C).($\overline{A}$+$\overline{B}$+$\overline{C}$)
    \item F=($\overline{A}$+$\overline{B}$+C).($\overline{A}$+B+C).(A+$\overline{B}$+C).(A+B+$\overline{C}$).(A+B+C)
\end{enumerate}

\section{Solution}

\subsection{Truth Table}
\begin{table}[h]
\centering
\begin{tabular}{|c|c|c|c|c|}
\hline
\textit{\textbf{A}} & \textit{\textbf{B}} & \textit{\textbf{C}} & \textit{\textbf{F}} & \textbf{Maxterms}                                     \\ \hline
0                   & 0                   & 0                   & 1                   & -                                                     \\
0                   & 0                   & 1                   & 0                   & \textit{A+B+$\overline{C}$}                           \\
0                   & 1                   & 0                   & 1                   & -                                                     \\
0                   & 1                   & 1                   & 0                   & \textit{A+$\overline{B}$+$\overline{C}$}              \\
1                   & 0                   & 0                   & 1                   & -                                                     \\
1                   & 0                   & 1                   & 0                   & \textit{$\overline{A}$+B+$\overline{C}$}              \\
1                   & 1                   & 0                   & 0                   & \textit{$\overline{A}$+$\overline{B}$+C}              \\
1                   & 1                   & 1                   & 0                   & \textit{$\overline{A}$+$\overline{B}$+$\overline{C}$} \\ \hline
\end{tabular}
\caption{\textit{Truth Table for eq.(1)}}
\label{tab:my-table}
\end{table}

From the above truth table, we get the following POS expression
\begin{equation}
    F=(A+B+\overline{C}).(A+\overline{B}+\overline{C}).(\overline{A}+B+\overline{C}).(\overline{A}+\overline{B}+C).(\overline{A}+\overline{B}+\overline{C})
\end{equation}
which is the same as option (C) above.

\subsection{K-map}
\begin{center}
\begin{karnaugh-map}[4][2][1][][]
    \maxterms{3,4,5,6,7}
    \minterms{0,1,2}
    \implicant{3}{7}
    \implicant{4}{6}
    % note: position for start of \draw is (0, Y) where Y is
    % the Y size(number of cells high) in this case Y=2
    \draw[color=black, ultra thin] (0, 2) --
    node [pos=0.7, above right, anchor=south west] {$AB$} % YOU CAN CHANGE NAME OF VAR HERE, THE $X$ IS USED FOR ITALICS
    node [pos=0.7, below left, anchor=north east] {$C$} % SAME FOR THIS
    ++(135:1);
\end{karnaugh-map}
\end{center}
the simplified POS expression obtained from the Karnaugh map above is
\begin{equation}
    F=(\overline{A}+\overline{B}).\overline{C}
\end{equation}

\subsection{Circuit Diagram}
\begin{center}
\begin{circuitikz}
    \draw
    (0,0) node[or port](myor1){}
    (2,1) node[and port](myand1){}
    
    (myor1.out)--(myand1.in 2)
    
    (myor1.in 1) node[anchor=east] {A}
    (myor1.in 2) node[anchor=east] {B}
    (myand1.in 1) node[anchor=east] {C}
    (myand1.out) node[anchor=west] {F}
    ;
    \node at (myor1.bin 1) [ocirc, left]{} ;
    \node at (myor1.bin 2) [ocirc, left]{} ;
    \node at (myand1.bin 1) [ocirc, left]{} ;
\end{circuitikz}
\end{center}
making a logic diagram using only NAND gates,
\begin{center}
\begin{circuitikz}
    \draw
    (0,0) node[nand port](mynand1){}
    (2,1) node[nand port](mynand2){}
    (0,2) node[nand port](mynand3){}
    (-2,2) node(C){C}
    (4,1) node[nand port](mynand4){}
    
    (C)--(mynand3.in 1)
    (C)--(mynand3.in 2)
    (mynand1.out)--(mynand2.in 2)
    (mynand3.out)--(mynand2.in 1)
    (mynand2.out)--(mynand4.in 1)
    (mynand2.out)--(mynand4.in 2)
    
    (mynand1.in 1) node[anchor=east] {A}
    (mynand1.in 2) node[anchor=east] {B}
    (mynand4.out) node[anchor=west] {F}
    ;
\end{circuitikz}
\end{center}

\end{document}
